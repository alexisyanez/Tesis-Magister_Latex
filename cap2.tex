\chapter{Resultados Pre-eliminares}
\lipsum[50-60]
2.	Marco Teórico

En este capítulo se presenta y definen los conceptos básicos que permiten el entendimiento de la materia investigada y del posterior trabajo realizado.

2.1.	Redes Digitales

A grandes rasgos corresponden  a la conexión  entre dispositivos electrónicos que utilizan un canal para poder intercambiar información relevante para cada uno de estos dispositivos. Esto constituye un sistema de comunicación que utiliza señales digitales y el cual presenta nuevos desafíos, como la discretización y cuantificación de las señales transmitidas, frecuencias de muestreo de señales, formas de hacer uso del canal físico, tecnologías de acceso al medio etc. 

2.2.	Redes Móviles
Estas redes son de tipo inalámbrico, es decir el canal físico corresponde al espectro radioeléctrico y requiere de antenas transmisoras y receptoras. Él objetivo de estas redes es que se pueda establecer comunicación con hosts que están constantemente cambiando su posición. 

2.2.1.	Redes Móviles Ad-hoc (MANET)
Estas redes se caracterizan por utilizar una arquitectura P2P, por  lo tanto, la información no viaja hacia un servidor , estableciéndose solo inter-conectividad entre los dispositivos. Las Redes Vehiculares Ad-hoc (VANET) corresponde a un tipo de estas redes  [19].

2.2.1.1.	Detección de Congestion de Trafico en             VANETs
Actualmente existen diversas aplicaciones que buscan resolver el problema de la congestión de tráfico a través de sistemas de VANETs. Éste se puede dividir en 3 grandes etapas. 1) Monitoreo de las principales variables que facilitan la detección de tráfico, 2) detección o predicción de congestión, y 3) Diseminación eficiente de la información [13]. 
Algunos autores (ver tabla 1) abordan las dos primeras etapas, mientras que otros desarrollan las  3 en plenitud, las cuales pueden  retro-alimentarse, pues, para detectar los nodos vecinos, utilizan una tabla en la cual se indexan la identificación, posición,  velocidad, etc. y en base a esto,  luego se puede diseminar la información de manera eficiente, utilizando solo ciertos nodos para retransmitir la información y no sobrecargar el canal de comunicación.
Además las diferentes propuestas se pueden diferenciar en el tipo de variable que utilizan para detectar la congestión de tráfico, algunos utilizan el tiempo de viaje en un segmento de la ruta, mientras que otros utilizan la posición y/o velocidad de vehículos en la vecindad. Estas diferencias se relacionan  con el tipo de arquitectura que utilizan los mecanismos, esta puede ser V2V, V2I-I2V.
Los sistemas V2V se pueden subdividir en aquellos que utilizan un sistema cooperativo distribuido a aquellos que utilizan uno centralizado o aquellos que utilizan un sistema de solicitudes. Las soluciones mas recientes introducen un sistema de diseminación multi-salto, es decir los nodos adquieren la información y la repiten a los nodos vecinos que están mas alejados de la fuente de información. Un resumen compacto del estudio de las propuestas se muestra en la Tabla 1.

		Caracteristicas
	Propuestas	Etapas	Arquitectura	Variable utilizada
[1]	Lakas (2009). 	1,2,3	V2V	Tiempo de Viaje
[2]	Mohandas (2009). 	1,2,3	V2I-I2V	- 
[3]	Bauza (2010). 	1 y 2	V2V	Velocidad y Densidad
[4]	Xu, Y. (2010). 	1,2,3	V2V,V2I-I2V	Velocidad y Densidad
[5]	Marfia (2011). 	1 y 2	I2V	Tiempo de Viaje
[6]	Singh  (2011). 	1,2,3	V2V	Velocidad y Posición
[7]	Zhang, (2011). 	1 y 2	-	Mapa de Velocidades
[8]	Terroso-sáenz (2012). 	1,2,3	V2V,V2I-I2V	Velocidad y Posición
[9]	Xu, Y. (2012).	1,2,3	I2V	Tiempo de Viaje
[10]	Bauza  (2013). 	1,2,3	V2V	Velocidad y Densidad
[11]	Martuscelli  (2013).	1,2,3	V2V	Posición
[12]	Younes  (2013). 	1,2,3	V2V	Posición
[13]	Gramaglia (2014). 	1,2,3	V2V	Posición
[14]	Milojevic (2014). 	1,2,3	V2V	Velocidad y Densidad
[15]	Shaikh (2014). 	1,2,3	V2V	Velocidad
[16]	Yuan  (2014). 	1,2,3	V2V	Velocidad y Densidad
[17]	Younes (2015). 	1,2,3	V2V	Velocidad
[18]	Turcanu  (2016). 	1,2,3	V2V,V2I-I2V	Posición

Tabla 1. Resumen de las propuestas estudiadas.

Tanto para detectar como para diseminar  los sistemas basados en beaconing, suelen sufrir desperfectos cuando se tiene congestión de tráfico, debido a la sobrecarga en el canal de comunicación lo que se conoce como tormenta de broadcast. por lo que saber, cuántos vecinos tiene un nodo, en tiempo real y de manera precisa, puede entregar beneficios tanto para la congestión de tráfico como para ajustar los mecanismos de recolección y diseminación de información, además de poder otorgar alternativas a los conductores que prefieren evadir el tráfico, lo puedan hacer manera eficiente. Esto presenta un gran desafío a resolver y explorar en las redes inter-vehiculares, debido a su alto impacto en los tiempos de detección.
